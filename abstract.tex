\begin{resumo}[Abstract]
 \begin{otherlanguage*}{english}
   The law enforcement agencies provide a wide variety of services to citizens, but when the need for a service, not always the public security organs are physically present, and the most efficient way to contact them is by emergencies phones they offer, then these phones should be immune to failure. So this work aims to show an improvement in the emergency call center Military Police of Dourados, therefore, is presented in conventional telephony concepts, VoIP (Voice over IP) and a hybrid tool because this will have to implement the functions of a traditional telephone exchange and also, VoIP protocols, and contain features of a PBX to solve technical problems of this telemarketing thus provide security for attendants and reliability for citizens in need these emergency services.

   \vspace{\onelineskip}
 
   \noindent 
   \textbf{Palavras-chaves}: \textit{VoIP. PSTN. Telemarketing. Aggregation Services Telephony.}
 \end{otherlanguage*}
\end{resumo}