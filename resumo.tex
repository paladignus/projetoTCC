\setlength{\absparsep}{18pt} % ajusta o espaçamento dos parágrafos do resumo
\begin{resumo}

Grandes mudanças, inovações e novas tecnologias em comunicação tem surgido nos últimos referente a telefonia, sendo que esta permitem uma vasta gama de serviços que podem ser agregados a telefonia convencional, entre todas podemos destacar a tecnologia de transmissão de Voz sobre IP, pois esta em constante inovação e expansão. A VoIP tem possibilitado uma maior flexibilidade no que se diz respeito a serviços referente a telefonia.

Indiscutivelmente que a abrangência da rede de telefonia convencional muito superior, pois possui uma estrutura instalada que envolve quase todas as localidades do planeta, porém a VoIP tem trazido significantes melhorias referente a transmissão de voz, e ainda conta com transmissão de mensagens de texto e até mesmo video.

Este trabalho tem como objetivo proporcionar uma melhoria no teleatendimento emergencial da Polícia Militar de Dourados, sendo que este trabalho consiste na investigação e apresentação de possíveis falhas técnicas ou mesmo humanas, mas para isso sera apresentados conceitos referentes à ambientes de telefonia, a fim de desenvolver um ambiente propício para protótipo funcional de um servidor PABX com emprego da tecnologia de transmissão de voz sobre IP, mais precisamente a ferramenta Asterisk e gama de serviços que esta possui, porém o foco principal é nas melhorias técnicas que esta ferramenta tem a proporcionar, e coerentemente trará benefícios nos recursos humanos deste teleatendimento.


 \textbf{Palavras-chaves}: \textit{VoIP. PSTN. Teleatendimento. Agregação de Serviços a Telefonia.}.
\end{resumo}
