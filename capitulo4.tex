%========================= CAPITULO 4 ==============================

\chapter{Aplicação} %\label{cap_exemplos}
%\thispagestyle{empty}
\section{Instalação do Asterisk}
A instalação do Asterisk requer e exige uma série de cuidados e detalhes,  para que se tenha um ambiente de telefonia estável e disponível o maior tempo possível. Logo, e necessário uma instalação onde busca-se a menor quantidade de problemas e com maior tempo operando, pois a compilação do Asterisk não é um processo trivial \cite{alexandrekeller2014}.

Apesar de haver duas formas de instalação, onde pode ser por SVN\footnote{Sistema de controle de versão} que mantêm um compartilhamento do desenvolvimento do sistemas com todas as novas funcionalidades, porém ainda nem sempre devidamente testadas, logo optou-se por baixar os pacotes compactados sendo eles:

\begin{itemize}
  \item dahdi-linux-current.tar.gz
  \item dahdi-tools-current.tar.gz
  \item libpri-1.4-current.tar.gz
  \item openr2-1.3.3.tar.gz
  \item libss7-1.0.2.tar.gz
  \item asterisk-13-current.tar.gz
\end{itemize}

E podem ser obtidos respectivamente nos seguintes endereços:

\begin{itemize}
  \item http://downloads.asterisk.org/pub/telephony/dahdi-linux/
  \item http://downloads.asterisk.org/pub/telephony/dahdi-tools/
  \item http://downloads.asterisk.org/pub/telephony/libpri/
  \item https://code.google.com/p/openr2/downloads/list
  \item http://downloads.asterisk.org/pub/telephony/libss7/
  \item http://downloads.asterisk.org/pub/telephony/asterisk/
\end{itemize}

Após o download dos mesmo foram descompactados e compilados na  mesma ordem em foram baixados, pois, os modulos são independentes, ou seja, a compilação de um módulo reflete diretamente na compilação do outro, a exemplo que caso seja compilado o módulo do Asterisk antes do módulo LIBPRI, a compilação do Asterisk não reconhecerá as funções habilitadas pelo pacote LIBPRI.

Após a instalação do Asterisk algumas pastas e arquivos são criados por ele para sua respectiva configuração e armazenamento das informações, pois o Asterisk é dividido em módulos, cada um representando uma funcionalidade como aplicação, função, canal de comunicação, protocolo e outros, e para correta configuração destas funcionalidades e necessário correta construção e configuração destes arquivos, e no decorrer do projeto, serão abordados os arquivos de configuração pertinentes para a correta construção do protótipo para ser testado no teleatendimento.

\subsection{Protótipo}
O protótipo consiste na construção de servidor PABX híbrido para o teleatendimento da Polícia Miltar de Dourados, onde será configurado as funcionalidades do Asterisk pertinentes a este teleatendimento, logo, será apresentado os arquivos com as configurações necessárias para esta finalidade.

\subsubsection{Plano de Discagem}
O plano de discagem define todo o funcionamento do servidor Asterisk, é onde define os grupos e regras de discagem, ou seja, como as chamadas de entradas e saída do servidor serão tratadas, e quais funcionalidades serão atividades e como funcionarão, o arquivo onde é programado o plano de discagem é o extensions.conf e fica localizado dentro da pasta /etc/asterisk/ \cite{alexandrekeller2014}.
Na figura 18 podemos observar como ficou sua configuração para o ambiente escolhido.